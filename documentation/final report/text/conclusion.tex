\section{Conclusion \& Future Work}\label{conclusion}

According to our discussion in previous sections, we can draw following conclusions:

\begin{itemize}
    \item   Many UFO sightings occur at night, between 9 PM and 12 PM during a day, no matter whether it is clear or partly-cloudy. 
    \item   Sighting distribution across U.S. are imbalanced. States that near lakes, deserts and oceans tend to report more UFO sightings. California reports the most sightings, which also gets highest estimated marginal means of sighting duration.
    \item   Whether an UFO sighting report is true or just a hoax has close relationship to numeric features we extract. By using location information, time of a day, UFO shapes, weather conditions, our models can detect fake reports with a relatively high accuracy.
    \item   For text features, the word "reptile" has the biggest positive score, while the word "meteor" has the biggest negative score. 
    \item   Total UFO sighting number increases as year goes by, although slightly decreased recently. By using population and year as independent variable, our regression model predicts that there maybe 5589 UFO sightings in 2017.
\end{itemize}

Many further work can be done on UFO sighting data research. First and foremost is to find a better way to label data. NUFORC labeled data manually based on other information. For example, if a sighting's description and environment conditions are similar to satellite launching, they may label it as a fake report. It is possible to build a machine learning model based on related news analysis, which could label UFO reports automatically and more accurately.

Another thing is that we believe UFO sightings are related to many other different factors, like economy and vegetation. Consider what we derived in section \ref{statistic}. We conclude that California has more UFO sightings than the other states because of topographical features of desert and coast. But is this the truth? Is it possible this conclusion results from it's great economy, or any other conditions that is unique for the Golden State? We don't know. Many other data need to be collected, some of which even need some field research. 

After all, we believe UFOs, or aliens, do exist. But finding out rules of their occurrence on Earth is really difficult. Hope our work on analyzing sightings across U.S. could help further studies on this field.




